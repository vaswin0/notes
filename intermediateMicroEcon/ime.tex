\documentclass{article}
\usepackage{amsmath,amssymb}


\title{Intermediate Microeconomics}

\begin{document}

The consumption set is denoted by $X$.
If $x$, $y$ $\in X$, then x and y are potential consumption bundles.
$x =(x_1, x_2,...,x_n)$ where $n$ is the number of goods and $x_k$ is the quantity of good $k$ in the consumer's consumption bundle. For consumer $i$, consumption set is denoted by $X_i$ and a consumption bundle $x_i = (x_{i1}, x_{12},..., x_{in})$


\section{Consumption Preference}

A preference relation $\succeq$ is an ordering over the elements of  $X$.
\\  $x \succeq y$ means " $x$ is atleast as good as  $y$". 
\\ Strict preference($\succ$):  $x \succ y$ means  $x \succeq y$ and       $y \not \succsim x$ i.e. "x is better than y"
\end{document}


