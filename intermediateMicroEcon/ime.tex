\documentclass{article}
\usepackage{amsmath,amssymb}
\usepackage{xcolor}

\title{Intermediate Microeconomics}

\begin{document}

The consumption set is denoted by $X$.
If $x$, $y$ $\in X$, then x and y are potential consumption bundles.
$x =(x_1, x_2,...,x_n)$ where $n$ is the number of goods and $x_k$ is the quantity of good $k$ in the consumer's consumption bundle. For consumer $i$, consumption set is denoted by $X_i$ and a consumption bundle $x_i = (x_{i1}, x_{12},..., x_{in})$


\section{Consumption Preference}



\begin{itemize}

	\item A preference relation $\succeq$ is an ordering over the elements of  $X$.
 $x \succeq y$ means " $x$ is atleast as good as  $y$ or $ x$ is weakly preferred over  $y$". 
	\item \textbf{ Strict preference($\succ$)}:  $x \succ y$ means  $x \succeq y$ ("$x$ is atleast as good as  $y$") and $y \not \succsim x$("$y$ is not atleast as good as  $x$) i.e. "$x$ is better than $y$" \\
	\item \textbf{Indifference} (\sim) : $x \sim y$ mean $x\succeqy$ and  $y \succeq x$ i.e "$x$ is just as good as $y$" 
	\item A preference relation $\succeq$ represents "preferences" of each individual.

		
\end{itemize}

\section{Axioms of Rational Choice}

 \begin{itemize}
		 \item \textbf{Completeness}: If $x$ and  $y$ are any two consumption possibilities, the consumer can always specify exactly one of the following possibilities.  \[
		 x \succ y, y \succ x, x \sim y
		 .\] \textcolor{red}{Any alternative can be compared}

 \item \textbf{ Transitivity}: $x \succeq y$ and  $ y \succeq z$  $\implies$  $ x \succeq z$ 
\\ \textcolor{red}{Choices must be internally consistent}
\item \textbf{Continuity}: If a consumer reports that $x \succ y$, then she must also report that  $x' \succ y$ for any  $y$ "close to" x.
		
		
\end{itemize}


\section{Utility}


\begin{itemize}
		\item A utility function $u:X \mapsto \mathbb{R} $ represents $\succeq$ if and only if for all  $x,y \in X$ 
				\[
				x \succeq y \Lonfleftrigharrow u(x) \geq u(y)
				.\] 
		\item The function is unique only up to an order-preserving transformation.
		\item Any preference relation that can be represented by a continuous function is rational.
\end{itemize}






\end{document}


